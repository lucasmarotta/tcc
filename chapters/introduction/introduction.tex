\chapter{Introdução}
\label{cap:introducao}

\begin{quotation}[]{Yoko Ono}
I'm so fast with technology. People think it all seems too much, but we'll get used to it. I'm sure it all seemed too much when we were learning to walk.
\end{quotation}

A expansão dos meios de comunicação através da Internet possibilitou o rápido acesso a todo tipo de informação de diversas áreas do mundo a todo lugar. Consumir conteúdo digital tornou-se atividade comum no dia das pessoas. Conforme mais se expande o acesso as mídias digitais mais conteúdo é gerado e mais está disponível para ler, ver, ouvir e interagir. Segundo \cite{Walker:2014} chegamos a uma era em que trafegamos uma quantidade enorme de dados que rapidamente perde-se a escala e cognição para o humano. Qual o significado de 400 milhões de tweets\footnote{Tweet é o nome utilizado para designar as publicações feitas na rede social do Twitter (https://www.merriam-webster.com/dictionary/tweet)} por dia? Usar o pensamento empírico de grandes matemáticos como “to measure is to know” (William Thomson) torna-se especialmente difícil com o volume de informações produzidas neste século. Com a quantidade de dados disponíveis não é irônico ouvir “não sei qual filme assistir”, pois apesar do fácil acesso existe uma grande sobrecarga a qual expõe o usuário a um mar de dados \citep{Wellman:BigData}, dificultando o acesso ao conteúdo que seja mais relevante.

O volume de informações apresenta-se como um obstáculo ao usuário que deseja consumir algum tipo conteúdo. Compras online possuem milhares de opções e nem todos estão dispostos a passar um grande tempo olhando o catálogo disponível. Uma das razões pela preferência de compra pela Internet é justamente a “falta de tempo”, conforme revela análise de \cite{Mykolas:2015a}. Dessa forma, é natural que o usuário recorra a alternativas para se guiar pelas informações e encontrar mais facilmente aquilo que lhe é mais útil. Para minimizar o obstáculo que o volume de informações se opõem, é comum apelar para ajuda de conhecidos, parentes, amigos, como apontado pela pesquisa de \cite{Mykolas:2015a}, onde um dos fatores relacionados ao consumidor que influenciam a opção pela compra pela Internet são as recomendações de outros usuários.

A larga difusão da Internet, principalmente pela Web, também cria um desafio pela busca de informação. Sistemas populares de recuperação de informação, como Google, amenizam o problema \citep{ISINKAYE2015261}, mas são deficientes quanto a personalização e priorização da informação em relação as preferências e interesses do usuário. Essa é uma das razões pelo grande aumento do desenvolvimento e procura por sistemas de recomendação. Sistemas de recomendação são sistemas de filtragem de itens que possuem objetivo de prever a avaliação e preferência do usuário \citep{Ricci2011}. Tais soluções contribuem ainda mais com a experiência do usuário no que diz ao conceito do \ac{NFC} que reflete na tendência de indivíduos em se engajar e aproveitar numa atividade \citep{Mykolas:2015a}. Esses sistemas filtram os dados para reduzir o problema da sobrecarga de informação \citep{Konstan2012}, podendo ser utilizados em diversos domínios como livros, filmes, músicas até para construir experiências em jogos online \citep{Activision:Glixel}.

Os sistemas de recomendação tipicamente possuem três tipos de abordagens para as sugestões: filtragem colaborativa, filtragem baseada em conteúdo e filtragem híbrida que leva em consideração as duas anteriores. Filtragem baseada em conteúdo são fundamentadas na descrição dos dados e nas preferências dos usuários \citep{Aggarwal2016}. Desse modo, o objetivo desse trabalho é construir um sistema de recomendação baseado em conteúdo, utilizando dados não estruturados para definir as preferências de usuários, de tal forma a explorar relações semânticas entre eles, criando novas possibilidades para recomendar novos itens.

\section{Motivação}
Na similaridade em termos associados aos itens de comparação, é comum em sistemas de recomendação para domínios como de livros e filmes seja comparado termos como gênero e autor. Nesse caso, pode-se analisar se já foi demonstrado interesse pelo usuário em filmes com esses termos, para que assim o sistema aprenda e recomende novos itens com essas mesmas características. Entretanto, pode ser interessante para o usuário encontrar filmes que não sejam necessariamente do mesmo gênero ou autor, mas que possuam narrativas mais similares ou relacionadas. Como exemplo considere os filmes \textit{Sucker Punch} \citep{SuckerPunch2011} e \textit{Labirinto do Fauno} \citep{LaberintoFauno2006}, possuem diferentes diretores e apesar de terem um tema de fantasia, se diferem bastante, pois o primeiro é um filme orientado para ação enquanto que o segundo é um drama que se passa num período de guerra. Na narrativa dos filmes é possível encontrar pontos de similaridade, como os dois tratarem de jovens garotas que entram num mundo fantasioso onde precisarão vencer uma série de desafios para superar dificuldades em tempos difíceis. Nesse sentido, analisar a similaridade do conteúdo da descrição de um filme que contenha um trecho da sua narrativa, pode levar ao usuário a sair do seu círculo tradicional de preferência, podendo contribuir com o NFC no uso de um sistema. Uma das propostas desse trabalho é explorar os resultados analisando a sinopse de filmes, buscando relações que vão além de uma relação sintática, mas que possuam um contexto semântico relacionável.

Para analisar a similaridade de filmes observando a descrição da narrativa, será utilizado um serviço presente na web semântica, o DBPedia\footnote{http://wiki.dbpedia.org}, para traçar formas de definir uma similaridade entre palavras presentes nas sinopses dos filmes, assim extraindo relações semânticas de termos presentes nos textos. A ideia é de que a análise entre termos possa se estender de uma comparação simples e sintática, mas para uma comparação que envolva-os em contexto cujo seja possível definir um relacionamento ontológico. Como exemplo simples, quando compara-se uma frase como "O homem comprou aquele carro", com "A mulher adquiriu aquela moto", nota-se que embora tenham palavras diferentes, são semelhantes, podendo intuitivamente traçar uma proximidade entre termos de cada uma. 

Como motivação para criar novas oportunidades e maneiras de descobrir filmes, sendo útil na busca por novos títulos pelos serviços na internet, um \ac{SR} torna-se uma ferramenta valiosa. Na concepção dessas recomendações, é necessário criar um perfil do usuário para sugerir novos itens, que para a proposta deste trabalho se basearão nas suas preferências, ou seja em outros filmes que demonstrou interesse. Dessa forma, o foco deste trabalho é explorar a similaridade entre textos pequenos, um dado não estruturado, através da análise da relação semântica de termos extraídos, buscando suas referências em entidades nos serviços de dados ligados na web semântica (apresentados no Capítulo \ref{cap:semantic_web}). O propósito dessa similaridade é para criar um sistema de recomendação, utilizando o domínio de filmes como ponto de interesse do público, embora as definições na construção desse sistema não sejam exclusivamente desenvolvidas para tal domínio.

\section{Problema}
O problema deste trabalho trata-se da deficiência e dificuldade quanto a sistemas de recomendação sugerir itens quando apenas o conteúdo sintático é analisado desprezando relações semânticas presentes do conteúdo. É tradicional construir um SR apenas observando as caraterísticas discretas dos itens, como propriedades e categorias, ou até de uma análise estatística, mas existe uma lacuna de informações que são desprezadas que podem ser extraídas analisando-as numa rede semântica de relações que as envolva.

Um problema também muito comum trata-se de como esses algoritmos de filtragem e personalização afetam as pessoas. O livro “The Filter Bubble” \cite{Pariser:2011} levanta preocupações sobre tais sistemas, onde o usuário fica fortemente sujeito a apenas ao mesmo tipo de conteúdo, ou informação que não venha criar conflitos de ponto de visão, o efeito bolha. Assim, utilizando um SR que apenas analisasse termos de gênero e título poderia deixar o usuário “preso” no círculo tradicional de preferência. Essa preocupação pode também ter um impacto negativo no sistema, já que é possível que os usuários viriam a encontrar outros conteúdos que poderiam ter interesse, mas são apenas encorajados àqueles mais tradicionais. 

Comumente a busca de informação em sistemas de recuperação, como o Google\footnote{https://www.google.com}, possui dados dispersos e por muitas vezes desorganizados, além da carência de dados personalizados e priorizados que considere os interesses do usuário para encontrar o item desejado. Somando a isso, propondo um sistema em que também seja possível extrair a similaridade da descrição das narrativas dos filmes, analisando e buscando outras relações semânticas com as entidades presentes, pode-se trazer resultados que amenizem o efeito bolha. Esse trabalho tem um dos objetivos de explorar que decorrências podem ser obtidas levando em consideração essa abordagem.

\section{Objetivos da Solução Proposta}

Este trabalho propõem a criação de um SR baseado em conteúdo que também utilize uma análise da similaridade semântica (ver capítulo \ref{cap:semantic_web}) entre os itens envolvidos. Para isso será proposto um modelo de recomendação que leve em consideração a descrição da narrativa do item. O objetivo é explorar que resultados podem ser obtidos realizando consultas ao serviço de dados ligados na web semântica, o DBPedia \footnote{http://wiki.dbpedia.org}. Para a construção do SR foi escolhido o domínio de filmes, como motivador e exemplo de aplicação que tire proveito desse sistema. Através de uma pequena análise empírica na rede de relacionamento do autor, percebeu-se que pessoas tendem a informar mais das preferências de filmes do que de livros, outro fator para a escolha do domínio.

Com o acesso a esse serviço da web semântica, serão analisadas entidades procurando ontologias e relações presentes nas sinopses dos filmes, através dos dados ligados na DBPedia. Assim, pode ser comparada à similaridade de dois filmes através da presença ou relação de ontologias extraídas dos termos das sinopses dos filmes. Como exemplo, caso um filme possua na sinopse o termo \textit{Morfeu} e o outro possua outras entidades sobre deuses mitológicos, como \textit{Zeus}, poderá ser criado um nível de similaridade e relevância entre as sinopses dos filmes.

Os dados dos filmes e de preferências dos usuários serão obtidos através de uma coleta do projeto MovieLens\footnote{https://movielens.org}, um banco de dados o qual possui 20 milhões de avaliações, por 138.000 usuários em 27.000 filmes, sendo comumente utilizados para construção de um \ac{SR}, o que facilita a realização de comparações. De posse dos dados serão definidos os modelos do usuário e dos itens para recomendação (ver \ref{sec:dataModel}), além de também propor uma nova métrica de similaridade semântica (ver \ref{ssec:sim_rec}) para a comparação de termos dos filmes e utilização no algoritmo de recomendação (ver \ref{ssec:rec_alg}).

\section{Estrutura}
Neste capítulo foi introduzido o problema e a motivação deste trabalho. Os próximos capítulos estão organizados da seguinte maneira: O Capítulo \ref{cap:recsys} apresenta os conceitos teóricos usados neste trabalho referentes a SR. O Capítulo \ref{cap:semantic_web} apresenta conceitos sobre a web semântica. O Capítulo \ref{cap:proposal} apresenta a proposta do SR com a resolução de um modelo de usuário que leve em consideração a descrição de itens, discutindo sua implementação. O Capítulo \ref{cap:evaluation} apresenta a avaliação do sistema, conclusões e considerações finais.