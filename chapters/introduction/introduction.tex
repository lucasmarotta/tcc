\chapter{Introdução}
\label{cap:introducao}

\begin{quotation}[]{Yoko Ono}
The computer is my favourite invention. I feel lucky to be part of the global village. I don't mean to brag, but I'm so fast with technology. People think it all seems too much, but we'll get used to it. I'm sure it all seemed too much when we were learning to walk.
\end{quotation}

A expansão dos meios de comunicação através da Internet possibilitou o rápido acesso a todo tipo de informação de diversas áreas do mundo a todo lugar. Consumir conteúdo digital tornou-se atividade comum no dia das pessoas. Conforme mais se expande o acesso as mídias digitais mais conteúdo é gerado e mais está disponível para ler, ver, ouvir e interagir. Segundo \cite{Walker:2014} chegamos a uma era em que trafegamos uma quantidade enorme de dados que rapidamente perde-se a escala e cognição para o humano. Qual o significado de 400 milhões de tweets\footnote{Tweet é o nome utilizado para designar as publicações feitas na rede social do Twitter (https://www.merriam-webster.com/dictionary/tweet)} por dia? Usar o pensamento empírico de grandes matemáticos como “to measure is to know” (William Thomson) torna-se especialmente difícil com o volume de informações produzidas neste século. Com a quantidade de dados disponíveis não é irônico ouvir “não sei qual filme assistir”, pois apesar do fácil acesso existe uma grande sobrecarga a qual expõe o usuário a um mar de dados \citep{Wellman:BigData}, dificultando o acesso ao conteúdo que seja mais relevante.

O volume de informações apresenta-se como um obstáculo ao usuário que deseja consumir algum tipo conteúdo. Compras online possuem milhares de opções e nem todos estão dispostos a passar um grande tempo olhando o catálogo disponível. Uma das razões pela preferência de compra pela Internet é justamente a “falta de tempo”, conforme revela análise de \cite{Mykolas:2015a}. Dessa forma, é natural que o usuário recorra a alternativas para se guiar pelas informações e encontrar mais facilmente aquilo que lhe é mais útil. Para minimizar o obstáculo que o volume de informações se opõem, é comum apelar para ajuda de conhecidos, parentes, amigos, como apontado pela pesquisa de \cite{Mykolas:2015a}, onde um dos fatores relacionados ao consumidor que influenciam a opção pela compra pela Internet são as recomendações de outros usuários.

A larga difusão da Internet, principalmente pela Web, também cria um desafio pela busca de informação. Sistemas populares de recuperação de informação, como Google, amenizam o problema \citep{ISINKAYE2015261}, mas são deficientes quanto a personalização e priorização da informação em relação as preferências e interesses do usuário. Essa é uma das razões pelo grande aumento do desenvolvimento e procura por sistemas de recomendação. Sistemas de recomendação são sistemas de filtragem de itens que possuem objetivo de prever a avaliação e preferência do usuário \citep{Ricci2011}. Tais soluções contribuem ainda mais com a experiência do usuário no que diz ao conceito do \ac{NFC} que reflete na tendência de indivíduos em se engajar e aproveitar numa atividade \citep{Mykolas:2015a}. Esses sistemas filtram os dados para reduzir o problema da sobrecarga de informação \citep{Konstan2012}, podendo ser utilizados em diversos domínios como livros, filmes, músicas até para construir experiências em jogos online \citep{Activision:Glixel}.

Os sistemas de recomendação tipicamente possuem três tipos de abordagens para as sugestões: filtragem colaborativa, filtragem baseada em conteúdo e filtragem híbrida que leva em consideração as duas anteriores. Filtragem baseada em conteúdo são fundamentadas na descrição dos dados e nas preferências dos usuários \citep{Aggarwal2016}. Desse modo, um dos objetivos deste trabalho é modelar um sistema com métricas que realizem a filtragem baseada em conteúdo, além de utilizar dados de serviços da web semântica para expandir as possibilidades de recomendação.

\section{Motivação}
Com a crescente popularização do acesso e uso da Web no mundo, cada vez é mais comum que pessoas escolham este ambiente para fazer compras, o comércio eletrônico. No Brasil, em 2015, movimentou R\$ 41,3 bilhões com o e-commerce\footnote{Modalidade de comércio que realiza suas transações financeiras por meio de dispositivos e plataformas eletrônicas (https://ecommercenews.com.br/o-que-e-e-commerce/)} segundo estudos da E-bit\footnote{https://www.ebit.com.br/} como aponta o \cite{Sebrae:2016}. O estudo também levanta que livros e revistas estão em 5º lugar como o tipo de item mais procurado. O crescimento do uso de dispositivos eletrônicos para realizar compras online, mostra que cada vez mais pessoas utilizam a Internet, especialmente para as redes sociais. Somente o Facebook\footnote{https://www.facebook.com} já registrou em 2017 2 bilhões de usuários ativos \citep{Statista:2017}. O tamanho da plataforma mostra que existe uma quantidade enorme de dados sobre usuários da Internet de todo o mundo, podendo ser fácil de encontrar preferências e relações de amizade. Esses dados servem como uma excelente fonte de busca para montar um perfil.

A grande quantidade de informação sobre os usuários presentes nessas redes sociais, é de amplo valor para construção de sistemas de recomendação. Em muitas dessas plataformas, é disponibilizado para terceiros uma \ac{API} para que, por exemplo, o usuário possa acessar uma aplicação de terceiros utilizando as credenciais dessa rede, o que pode facilitar a adesão de novos serviços. Assim, é possível construir um sistema de recomendação baseado em conteúdo já com uma infraestrutura de dados conhecida e amplamente difundida e aceita pelos usuários. A utilização do \ac{SR} com filtragem baseada em conteúdo, aprende e recomenda itens que sejam similares aos que o usuário já demonstrou interesse \citep{Ricci2011}.

Na similaridade em termos associados aos itens em comparação, é comum em domínios como de livros e filmes seja comparado termos como gênero e autor. Nesse caso, é analisando se já foi demonstrado interesse em filmes com esses termos, para que assim o sistema aprenda e recomende novos filmes com esses mesmos. Entretanto, pode ser interessante para o usuário encontrar filmes que não sejam necessariamente do mesmo gênero ou autor, mas que possuam narrativas mais similares ou relacionadas. Como exemplo considere os filmes \textit{Sucker Punch} \citep{SuckerPunch2011} e \textit{Labirinto do Fauno} \citep{LaberintoFauno2006}, possuem diferentes diretores e apesar terem um tema de fantasia, se diferem bastante pois, o primeiro é um filme mais de ação enquanto que o segundo é um drama que se passa num período de guerra. Na narrativa dos filmes é possível encontrar novos pontos de similaridade, como os dois tratarem de jovens garotas que entram em um mundo fantasioso que irão precisar vencer uma série de desafios para superar dificuldades em tempos difíceis. Nesse sentido, analisar a similaridade de conteúdo da descrição de um filme que contenha um trecho da sua narrativa, pode levar ao usuário a sair do seu círculo tradicional de preferência, podendo contribuir com o NFC no uso de um sistema. Uma das propostas desse trabalho é explorar os resultados analisando a sinopse de filmes, como uma síntese da narrativa.

Além de analisar a similaridade de filmes também observando a descrição da narrativa, será utilizado o serviço da web semântica DBPedia\footnote{http://wiki.dbpedia.org}, para obter mais informações das descrições dos filmes extraindo relações semânticas de entidades presentes nos textos. Para o SR prover as informações personalizadas é necessário criar um perfil do usuário para indicar o tipo de conteúdo, baseando-se em itens que sejam similares que aos que usuário gostou no passado. Expandindo o alcance do SR será proposto e avaliado. utilizando o domínio de filmes, um modelo que leve em consideração nas métricas de avaliação, a relação semântica das entidades presentes nas descrições das narrativas. O objetivo é explorar o relacionamento das ontologias presentes na sinopse do filme, pelos dados ligados (apresentados no Capítulo \ref{cap:semantic_web}) oferecidos no serviço da web semântica. Com os dados ligados é possível estabelecer uma relação entre diferentes fontes de dados para formar um único espaço global. É importante ressaltar que os trabalhos apresentados aqui não possuem características especificamente voltadas ao domínio de filmes, mas este é apenas usado como motivador para criação de um SR.

\section{Problema}
O problema deste trabalho trata-se da deficiência e dificuldade quanto a sistemas de recomendação sugerir itens quando apenas o conteúdo sintático é analisado desprezando relações semânticas presentes do conteúdo. É tradicional construir um SR apenas observando as caraterísticas discretas dos itens, como propriedades e categorias, mas existe uma lacuna de informações que são desprezadas que podem ser extraídas analisando-as numa rede semântica de relações que as envolva.

Um problema também muito comum ao montar o perfil do usuário, é a falta de informação sobre suas preferências. O sistema ainda não obteve interações suficientes para montar um perfil, afetando diretamente a qualidade das recomendações. Com o serviço do Facebook \footnote{https://www.facebook.com} existe a possibilidade de extrair dados das preferências para um grande número de pessoas de forma automática e transparente, uma vez que já é amplamente aceito pelos usuários. Dessa forma, além de facilitar a montagem do perfil do usuário, de imediato diminui a sobrecarga de informação que ainda passaria para poder usufruir de um SR.

Outra questão trata-se de como esses algoritmos de filtragem e personalização afetam as pessoas. O livro “The Filter Bubble” \cite{Pariser:2011} levanta preocupações sobre tais sistemas, onde o usuário fica fortemente sujeito a apenas ao mesmo tipo de conteúdo, ou informação que não venha criar conflitos de ponto de visão, o efeito bolha. Assim, utilizando um SR que apenas analisasse termos de gênero e título poderia deixar o usuário “preso” no círculo tradicional de preferência. Essa preocupação pode também ter um impacto negativo no sistema, já que é possível que os usuários venham a encontrar outros conteúdos que poderiam ter interesse, mas são apenas encorajados a aqueles mais tradicionais. 

A busca tradicional de informação em sistemas de recuperação, como o Google\footnote{https://www.google.com}, possui dados dispersos e por muitas vezes desorganizados, além da carência de dados personalizados e priorizados que considere os interesses do usuário para encontrar o item desejado. Somando a isso, propondo um sistema em que também seja possível extrair a similaridade da descrição das narrativas dos filmes, analisando e buscando outras relações semânticas com as entidades presentes, podem trazer resultados que amenizem o efeito bolha. Esse trabalho tem um dos objetivos de explorar que decorrências podem ser obtidas levando em consideração essa abordagem.

\section{Objetivos da Solução Proposta}

Este trabalho propõem a criação de um SR baseado em conteúdo que também utilize uma análise da similaridade semântica (ver capítulo \ref{cap:semantic_web}) entre os itens envolvidos. Para isso será proposto um modelo de usuário que leve em consideração a descrição da narrativa do item. O objetivo é explorar que resultados podem ser obtidos realizando consultas ao serviço DBPedia \footnote{http://wiki.dbpedia.org}. Para a construção do SR foi escolhido o domínio de filmes, como motivador e exemplo de aplicação que tire proveito desse sistema. Através de uma pequena análise empírica na rede de relacionamento do autor, percebeu-se que as pessoas tendem a informar mais das preferências de filmes do que de livros, outro fator para a escolha do domínio.

Com o acesso a esse serviço da web semântica, serão analisadas entidades procurando ontologias e relações presentes nas sinopses dos filmes, através dos dados ligados na DBPedia. Assim, pode ser comparada à similaridade de dois filmes através da presença ou relação de ontologias presentes na descrição. Como exemplo, caso um filme possua na sinopse o termo \textit{Morfeu} e o outro não, mas possua outras entidades sobre deuses mitológicos, como \textit{Zeus}, poderá ser criado um nível de similaridade e relevância com o novo filme.

Os filmes de preferência do usuário serão obtidos pelo Facebook\footnote{https://www.facebook.com} ou registrando pela própria plataforma do protótipo apresentado neste trabalho. Inicialmente o usuário se registrará na aplicação desenvolvida por este trabalho, em seguida coletará as informações referentes a filmes que ele avaliou ou tenha preferência. Haverá também previamente uma coleta de dados sobre filmes através do projeto MovieLens\footnote{https://movielens.org}, um banco de dados o qual possui 20 milhões de classificações e 465.000 aplicações de tags a 27.000 filmes por 138.000 usuários (ver \ref{sec:dataModel}), para facilitar a obtenção de dados dos a fim de relacionar as preferências do usuário aos filmes. Por fim o sistema modelará um perfil do histórico de suas preferências a fim de recomendar novos itens, realizando comparações com a métrica de similaridade proposta (ver \ref{ssec:sim_rec}) que serão avaliados pelo algoritmo de recomendação, fornecendo os novos itens.

\section{Estrutura}
Neste capítulo foi introduzido o problema e a motivação deste trabalho. Os próximos capítulos estão organizados da seguinte maneira: O Capítulo \ref{cap:recsys} apresenta os conceitos teóricos usados neste trabalho referentes a SR. O Capítulo \ref{cap:semantic_web} apresenta conceitos sobre a web semântica. O Capítulo \ref{cap:proposal} apresenta a proposta do SR com a resolução de um modelo de usuário que leve em consideração a descrição de itens, discutindo sua implementação. O Capítulo \ref{cap:evaluation} apresenta a avaliação do sistema, conclusões e considerações finais.