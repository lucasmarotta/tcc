Por facilitar a procura por informação no mar de dados da Internet, Sistemas de Recomendação (SR) são extremamente populares na Web. Usualmente SR tentam prever avaliações de usuários sobre um item desconhecido, para gerar recomendações personalizadas. Nesses sistemas, em especial os baseados em conteúdo, características de itens são processadas para identificar outros itens relacionados, mas é comum que sejam negligenciadas relações semânticas entre eles. Focar unicamente em dados sintáticos, favorece o efeito bolha, o que é caracterizado por usuários sendo menos expostos a itens relevantes e inesperados, algo desejável num SR. Encontrar itens com similaridades semânticas pode minimizar esse efeito, já que provê uma ainda relevante, porém mais abrangente similaridade. Nesse sentido, este trabalho propõe um Sistema de Recomendação, baseado em conteúdo, com Similaridade Semântica Ponderada por Links de Recursos (RLWS) na \textit{DBPedia}. O objetivo é verificar que resultados são obtidos pela comparação de termos da sinopse dos filmes, onde RLWS analisa relações semânticas diretas e indiretas entre eles, usando o DBPedia. Sendo assim, foi conduzido um experimento comparando RLWS com a conhecida similaridade do cosseno. Considerando um conjunto de cinco itens ($k=5$), o sistema proposto melhorou a precisão média (MAP) em 51\%, quando privilegiado relacionamentos indiretos, e 27\% para os diretos. Além disso, a proposta também melhorou o desempenho da métrica MRR em 26\% privilegiando relacionamentos indiretos, e de 11\% para diretos.

\begin{keywords}
	Sistema de Recomendação, Recomendação Baseada em Conteúdo, Web Semântica, Similaridade Semântica
\end{keywords}