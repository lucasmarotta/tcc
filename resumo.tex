Para minimizar o obstáculo da procura por informação no mar de dados da Internet, é comum a utilizar sistemas de recomendação. Nesses sistemas itens são filtrados com o objetivo prever a avaliação dos usuários, caracteristicamente divididos entre aqueles que se baseiam no histórico de preferências, ou daqueles que utilizam avaliações de outros usuários. Para esses sistemas é definido uma métrica de similaridade capaz de comparar itens, onde domínios de filmes e livros, tipicamente utilizam características de itens como gênero, autor. Entretanto, pode ser relevante encontrar itens com descrições ou narrativas mais similares, podendo diminuir o \enquote{efeito filtro bolha}, onde o usuário fica fortemente exposto ao mesmo tipo de conteúdo a não criar conflitos de pontos de visão. Dessa forma, este trabalho propõem um Sistema de Recomendação com Similaridade Semântica Ponderada por Links de Recursos na \textit{DBPedia}, utilizando o domínio de filmes como exemplo. Além de tratar a deficiência de sugerir itens quando apenas o conteúdo sintático é analisado, o objetivo é verificar quais resultados são obtidos com uma comparação entre termos da sinopse dos filmes, estendendo para uma análise que utiliza relações semânticas. Para comparar tais palavras propõe-se uma nova similaridade semântica verificando os relacionamentos diretos e indiretos entre tais palavras presentes no serviço da Web Semântica, o \textit{DBPedia}. Nos resultados a nova medida foi comparada com a similaridade do cosseno que utiliza. Os experimentos confirmaram ser possível tirar proveito de relações semânticas para recomendar, obtendo bons resultados. A nova medida obteve uma precisão $p@5$ 51\% superior pela métrica de avaliação \textit{MAP}, priorizando relacionamentos indiretos entre termos, e de 27\% para diretos. Já para a métrica \textit{MRR} conferiu um resultado nos 5 primeiros itens, 26\% superior para relacionamentos indiretos e de 11\% para diretos.

\begin{keywords}
	Sistema de Recomendação, Recomendação Baseada em Conteúdo, Web Semântica, Similaridade Semântica
\end{keywords}