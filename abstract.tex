For minimize the obstacle to search information in the Internet sea of data, is common to utilize recommender systems. On these systems items are filtered with the goal to predict the user evaluation, beign caractheristically divided between those based on the history of preferences, and those which use the evaluations of another users. To theses systems is defined a similarity metric capable to compare itens, where domains like films and books, typically uses features of itens like genre, author to recommend. Nonetheless, can be relevant find itens with descriptions and narratives more similar, allowing the diminish of the "bubble filter effect", where the user stay strongly exposed to the same kind of content, to not create conflicts of view points. Thus,  this work  proposes a Recommender System with Semantic Similarity Weighted by Links of Resources on DBPedia, utilizing the film domain as a example. Beyond treat the deficiency to sugest itens when only the sintatical content is analysed, the objective is to verify which results are obtainable with a comparison between terms found at film sinopses, extending to a analysis that uses semantic relations. To compare those words is proposed a new semantic similarity to check the direct and indirect relationships between those terms in a service of Semantic Web, the \textit{DBPedia}. At the results the new metric was compared to cosine similarity. So, the experiments confirmed that is possible to take advantage of semantic relationships to recommend, obtaining good results. The new measure got a precision $p@5$ 51\% superior by the evaluation metric \textit{MAP}, prioritizing indirect relationships between terms, and 27\% superior for the direct ones. Meanwhile, to the metric \textit{MRR}, was checked at the first 5 itens a result 26\% superior for indirect relationships and 11\% for the direct ones.

\begin{keywords}
	Recommender Systems, Content-Based Recomendation, Semantic Similarity, Semantic Web
\end{keywords}