With the goal to facilitate the efforts when searching information on the Web, Recommender Systems (RS) have become extremely popular in recent years on the Web. Usually, RS try to predict the user’s evaluation over an unknown item to generate personalized recommendations. Those systems, especially those content-based, process syntactic data (e.g., item features) to identify new related items, but often neglect the semantic similarities between them. Focusing only on syntactic data favors the “bubble filter effect” - an effect characterized by the user not being exposed to unexpected and relevant items, a desired feature for RS. Finding items with semantic similarities minimizes the “bubble filter effect” since it can provide a broader and more relevant similarity. In this sense, this work proposes a Recommender System (content-based) with a Resource Link-Weighted Similarity (RLWS), using \textit{DBPedia}. The proposed system verifies which results are obtainable by comparing terms from film synopses, and then RLWS analyses the direct and indirect semantic relations between them, using the DBPedia. We conduct an experimental evaluation comparing the RLWS with the well-known cosine similarity. Considering a result set of five items ($k=5$), the proposed system improves the MAP performance by 51\% when weighting more indirect relationships between terms, and for the direct relationships by 27\%. In addition, the proposal improves the MRR performance in 26\% weighting more indirect relationships, and 11\% using the direct ones.

\begin{keywords}
	Recommender Systems, Content-Based Recomendation, Semantic Similarity, Semantic Web
\end{keywords}